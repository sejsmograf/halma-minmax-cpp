\documentclass[a4paper, 12pt]{article}
\usepackage[hmargin=2cm,vmargin=2.5cm]{geometry}
\usepackage[polish]{babel}
\usepackage{listings}
\usepackage{color}
% Define Gruvbox colors
\definecolor{gruvbox-bg}{rgb}{0.156,0.156, 0.156}
\definecolor{gruvbox-fg}{rgb}{0.92 , 0.85, 0.69}
\definecolor{gruvbox-comment}{rgb}{0.57, 0.51, 0.45}
\definecolor{gruvbox-purple}{rgb}{0.82, 0.52, 0.60}
\definecolor{gruvbox-green}{rgb}{0.72, 0.72, 0.14}

% Define the lstlisting style using Gruvbox colors
\lstdefinestyle{mystyle}{
    language=c++,
    backgroundcolor=\color{gruvbox-bg},   
    basicstyle=\color{gruvbox-fg}\ttfamily\scriptsize,
    commentstyle=\color{gruvbox-comment},
    keywordstyle=\color{gruvbox-purple},
    stringstyle=\color{gruvbox-green},
    numberstyle=\tiny\color{gruvbox-comment},
    breakatwhitespace=false,         
    breaklines=true,                 
    captionpos=b,                    
    keepspaces=true,                 
    numbers=left,                    
    numbersep=5pt,                  
    showspaces=false,                
    showstringspaces=false,
    showtabs=false,                  
    tabsize=2
}

\lstset{style=mystyle}


\usepackage{float}
\usepackage{enumerate}
\usepackage{makecell}
\usepackage{graphicx}
\usepackage{hyperref}
\usepackage{amsmath}
\usepackage[table]{xcolor}
\usepackage{tabularray}
\usepackage{csquotes}

\usepackage[T1]{fontenc}



\setlength{\arrayrulewidth}{0.4mm}

\newcommand{\HRule}{\rule{\linewidth}{0.5mm}}


\usepackage{enumitem}

\begin{document}

\begin{titlepage}
  \begin{center}
    % logo
    \includegraphics[width=0.5\textwidth]{images/logo.png}~\\[1cm]

    \textsc{Sprawozdanie z zajęć laboratoryjnych\\ Sztuczna inteligencja i inżynieria wiedzy\\[3cm]}


    \HRule \\[0.4cm]
    {\large \bfseries Lista 02 \\ Minmax, Alphabeta \\[0.4cm]}
    \HRule \\[8cm]


    Tomasz Mroczko, 266604 \\[3cm]

    \vfill

    {\today}

  \end{center}
\end{titlepage}


\newpage
\tableofcontents
\newpage

  % \begin{figure}[H]
  %   \centering
  %   \includegraphics[width=0.8\textwidth]{2024-04-07-13-00-26.png} 
  % \end{figure}

\section{Wprowadzenie}
Celem projektu było zaimplementowanie algorytmu 
\textit{Minmax} oraz jego ulepszonej wersji \textit{Alphabeta} oraz zastosowanie 
ich do gry w \textit{Halma}. Zadanie wykonane zostało najpierw w języku \textit{Python},
jednak ze względu na interpretowany charakter tego języka oraz jego słabą wydajność, 
zdecydowano się na przepisanie algorytmów do języka \textit{C++}.



\section{Zadania}

\subsection{Zdefiniowanie stanu gry i funkcji generującej możliwe ruchy dla gracza}
\subsubsection{Stan gry}
Stan gry jest reprezentowany przez obiekt klasy \textit{Board},
która przechowuje tablicę dwuwymiarową \textit{16x16} reprezentującą planszę do gry.
Każde pole planszy jest reprezentowane przez jedną z wartości 
enumeracji \textit{FieldType}
\begin{lstlisting}
enum FieldType {
	EMPTY,
	WHITE,
	BLACK
};
\end{lstlisting}
\vspace{0.5cm}
\hrule
\vspace{1cm}


Deklaracja klasy \textit{Board} wygląda następująco
\begin{lstlisting}
class Board {
public:
    static const int BOARD_SIZE = 16;
    static const int PIECE_COUNT = 19;
    static const vector<pair<int, int>> DIRECTIONS;
    static const unordered_map<FieldType, vector<pair<int, int>>> PLAYER_CAMPS;

    Board();
    vector<piece_move> getPlayerMoves(FieldType playerType) const;
    vector<pair<int, int>> getPlayerPositions(FieldType playerType) const;
    vector<pair<int, int>> getPlayerGoalCamp(FieldType playerType) const;
    int playerPiecesInGoalCamp(FieldType playerType) const;
    pair<int, int> getEmptyGoal(vector<pair<int, int>> playerGoalCamp) const;
    void makeMove(piece_move move);
    void undoMove(piece_move move);
    void printBoard() const;

private:
    FieldType state[BOARD_SIZE][BOARD_SIZE];
    vector<piece_move> getPieceMoves(int row, int col) const;
    vector<piece_move> getDirectMoves(int row, int col) const;
    vector<piece_move> getJumpMoves(int row, int col) const;
    bool isWithinBounds(int row, int col) const;
    void initializeDefaultBoard();
};
\end{lstlisting}
Statyczne pola klasy przechowują informacje o 
rozmiarze planszy, liczbie pionków, kierunkach ruchu oraz
pozycjach obozów graczy. Metody klasy zostaną opisane w dalszej
części sprawozdania. 
Stan planszy przechowywany jest w tablicy dwuwymiarowej \textit{state}.
\vspace{0.5cm}
\hrule
\vspace{1cm}

\subsubsection{Metoda genrująca ruchy dla gracza}
Ruchu danego gracza generowane sa przez metodę 
\textit{Board.getPlayerMoves(FieldType playerType)}.
\begin{lstlisting}
vector<piece_move> Board::getPlayerMoves(FieldType playerType) const {
    vector<pair<int, int>> playerPositions = getPlayerPositions(playerType);
    vector<piece_move> allMoves;

    for (const pair<int, int> &position : playerPositions) {
        int row = position.first;
        int col = position.second;

        vector<piece_move> moves = getPieceMoves(row, col);
        allMoves.insert(allMoves.end(), moves.begin(), moves.end());
    }
    return allMoves;
}
\end{lstlisting}
Metoda ta pobiera pozycje wszystkich pionków gracza, 
a następnie dla każdego z nich generuje możliwe ruchy
za pomocą metody \textit{Board.getPieceMoves(int row, int col)}.
\vspace{0.5cm}
\hrule
\vspace{1cm}

Metoda \textit{Board.getPieceMoves(int row, int col)} 
generuje możliwe ruchy dla pionka na pozycji \textit{(row, col)}.
Wywołuje ona metody \textit{Board.getDirectMoves(int row, int col)} oraz
\textit{Board.getJumpMoves(int row, int col)}, a następnie 
łączy wyniki tych metod.
\begin{lstlisting}
vector<piece_move> Board::getPieceMoves(int row, int col) const {
    vector<piece_move> moves = getDirectMoves(row, col);
    vector<piece_move> jumpMoves = getJumpMoves(row, col);
    for (piece_move move : jumpMoves) {
        moves.push_back(move);
    }
    return moves;
};
\end{lstlisting}
\vspace{0.5cm}
\hrule
\vspace{1cm}

Generowanie bezpośrednich ruchów realizoawnane jest przez metodę
\textit{Board.getDirectMoves(int row, int col)}.
\begin{lstlisting}
vector<piece_move> Board::getDirectMoves(int row, int col) const {
    vector<piece_move> directMoves;
    for (pair<int, int> direction : DIRECTIONS) {
        int adjRow = row + direction.first;
        int adjCol = col + direction.second;
        if (isWithinBounds(adjRow, adjCol) &&
            state[adjRow][adjCol] == FieldType::EMPTY) {
            piece_move move = {row, col, adjRow, adjCol};
            directMoves.push_back(move);
        }
    }
    return directMoves;
};
\end{lstlisting}
Dla każdego z kierunków ruchu sprawdzane jest czy pole jest puste.
Jeśli tak, to zapisywany jest prawidłowy ruch.


\vspace{0.5cm}
\hrule
\vspace{1cm}
Generowanie ruchów możliwych do wykonania poprzez skakanie 
realizowane jest przez metodę \textit{Board.getJumpMoves(int row, int col)}.
\begin{lstlisting}
vector<piece_move> Board::getJumpMoves(int row, int col) const {
    vector<piece_move> jumpMoves;

    queue<pair<int, int>> toVisit;
    set<pair<int, int>> visited;

    toVisit.push({row, col});
    visited.insert({row, col});

    while (!toVisit.empty()) {
        pair<int, int> currPos = toVisit.front();
        toVisit.pop();

        for (pair<int, int> direction : DIRECTIONS) {
            int adjRow = currPos.first + direction.first;
            int adjCol = currPos.second + direction.second;
            if (!isWithinBounds(adjRow, adjCol) ||
                state[adjRow][adjCol] == FieldType::EMPTY) {
                continue;
            }
            int jumpRow = adjRow + direction.first;
            int jumpCol = adjCol + direction.second;

            if (!isWithinBounds(jumpRow, jumpCol) ||
                state[jumpRow][jumpCol] != FieldType::EMPTY) {
                continue;
            }

            if (!visited.contains({jumpRow, jumpCol})) {
                visited.insert({jumpRow, jumpCol});
                toVisit.push({jumpRow, jumpCol});
                piece_move move = {row, col, jumpRow, jumpCol};
                jumpMoves.push_back(move);
            }
        }
    }
    return jumpMoves;
};
\end{lstlisting}
Metoda ta korzysta z algorytmu BFS z dodatkowymi ograniczeniami,
w celu odwiedzenia wszystkich możliwych pól na które można skoczyć.
Dla każdego z kierunków, sprawdzane jest czy pole jest zajęte
oraz czy pole za nim jest puste. Jeśli tak, to do kolejki 
dodawane jest to pole, oraz ruch jest zapisywany jako możliwy.
Dopóki kolejka nie jest pusta, algorytm kontynuuje przeszukiwanie.


Warto zaznaczyć że ciekawym i naturalnym 
podejściem do generowanie ruchów jest także podejście rekurencyjne.
W tym przypadku, z moich obserwacji wynika, że podejście iteracyjne 
jest jednak bardziej wydajne oraz bardziej naturalne w kontekście
przeszukiwania możliwych ruchów w grze planszowej.
\vspace{0.5cm}
\hrule
\vspace{1cm}



\subsubsection{Klasa reprezentująca grę}
Dodatkowo, klasą która posiada obiekt klasy \textit{Board} oraz
metody do przeprowadzania gry jest klasa \textit{Halma}.
\begin{lstlisting}
class Halma {
public:
    static const FieldType PLAYER_ONE = FieldType::WHITE;
    static const FieldType PLAYER_TWO = FieldType::BLACK;
    Halma();
    void printBoard() const;
    void switchTurn();
    void makeMove(piece_move move);
    void makeMockMove(piece_move move);
    void undoMockMove(piece_move move);
    bool checkWinner(FieldType playerType);
    void gameFinished();
    Board &getBoard();
    FieldType getCurrentPlayer();
    bool isGameOver;
private:
    FieldType currentPlayer;
    Board board;
};
\end{lstlisting}
Klasa ta udostępnia metody do przeprowadzenia oraz symulowania gry. 
Metody \textit{makeMockMove} oraz \textit{undoMockMove} służą do
symulowania ruchów oraz ich cofania. Modyfikują one stan planszy,
jednak zawsze zostają wycofane. Pozwala to na przeprowadzanie 
symulacji ruchów w algorytmach \textit{Minmax} oraz \textit{Alphabeta}, 
bez konieczności kopiowania planszy, co pozwala na optymalizację 
zużycia pamięcia oraz (najpradowpodobniej) czasu.


\subsection{Zbudowanie zbioru heurystyk - 3 różne strategie}
\subsubsection{Struktura programu} 
Program gwarantuje klasę czysto wirtualną (interfejs) \textit{HalmaPlayer},
który posiada jendą metodę \textit{makeMove}. Gracze 
dziedziczą po tej klasie, implementując swoje strategie.
\begin{lstlisting}
class HalmaPlayer {
public:
    virtual void makeMove(Halma &game) = 0;
};
\end{lstlisting}


Kolejny interfejs \textit{BoardEvaluator} posiada metodę \textit{evaluateBoard}.
Za pomocą tego interfejsu realizowane są różne oceny stanu planszy, 
realizowane następnie przez graczy implementujących interfejs \textit{HalmaPlayer}.
\begin{lstlisting}
class BoardEvaluator {
public:
    virtual float evaluateBoard(const Board &board, FieldType player) const = 0;
};
\end{lstlisting}
Metoda przyjmuje referencję do planszy gry oraz kolor gracza, którego pozycja 
jest ewaluowana.

Ostanim interfejsem używanym do oceny stanu gry jest \textit{PawnHeuristic}.
Pozwala on ocenić odległość pojdeynczego pionka od celu.
\begin{lstlisting}
class PawnHeuristic {
public:
    virtual float evaluatePawnScore(int row, int col, int goalRow,
                                    int goalCol) const = 0;
};
\end{lstlisting}


\subsubsection{Heurystyki}
\begin{itemize}
  \item \textbf{Heurystyka 1} - Odległość Manhattan od celu w obozie przeciwnika\\
  Pierwsza heurestyka polega na ocenie odległości pionków od celu.
  W każdej ocenie planszy, wybierany jest cel pionków. 
  Celem jest pierwsze puste pole w obozie przeciwnika.
  Ocena jest obliczana następująco:
  dla każdego pionka obliczana jest odległość Manhattan od celu.
  Następnie jest ona podnoszona do kwadratu (w celu "zachęty" do opuszczenia obozu).
  Od ostatecznej oceny odejmowana jest ta odległość (im mniejsza, tym lepiej).
  Dodatkowo, mocno premiowane są pionki w obozie przeciwnika. W celu 
  zwiększenia różnorodności gry oraz zniwelowania ryzyka zablokowania 
  pionków w cyklu, ocena jest modyfikowana przez losową wartość z przedziału (0, 2).
\begin{lstlisting}
CornerDistance::CornerDistance(const PawnHeuristic &pawnHeuristic)
    : pawnHeuristic(pawnHeuristic){};

float CornerDistance::evaluateBoard(const Board &board,
                                    FieldType player) const {

    vector<pair<int, int>> goalCamp = board.getPlayerGoalCamp(player);
    vector<pair<int, int>> playerCamp = board.PLAYER_CAMPS.at(player);
    pair<int, int> goalPosition = board.getEmptyGoal(goalCamp);
    vector<pair<int, int>> playerPositions = board.getPlayerPositions(player);
    float evaluation = 0;

    for (const pair<int, int> &position : playerPositions) {
        int row = position.first;
        int col = position.second;
        float penalty = pawnHeuristic.evaluatePawnScore(
            row, col, goalPosition.first, goalPosition.second);
        evaluation -= squareDistancePenalty(penalty);

        bool isInGoalCamp = false;
        for (const auto &goal : goalCamp) {
            if (goal == position) {
                isInGoalCamp = true;
                break;
            }
        }
        if (isInGoalCamp) {
            evaluation += 1000;
        }
        evaluation += dis(gen);
    }

    return evaluation;
};

float CornerDistance::squareDistancePenalty(float penalty) const {
    return penalty * penalty;
}
\end{lstlisting}
Powyżej znajduje się kod źródłowy klasy \textit{DistanceEvaluator}.
Pierwsza heurestyka, to po prostu utworzenie jej instancji 
z odpowiednią heurystyką dla pionków.
\begin{lstlisting}
const PawnHeuristic &distance = ManhattanDistance();
CornerDistance evaluator(distance);
\end{lstlisting}
\item \textbf{Heurystyka 2} - Odległość od celu w obozie przeciwnika, z uwzględnieniem 
odległości od środka planszy\\
W tej heurystyce, pod uwagę brana jest zarowno odległość od celu w obozie przeciwnika,
jak i odległość od środka planszy. Odległość od środka planszy niest jest 
tak mocno premiowana jak odległość od celu, jednak nakłania ona pionki do 
pozostawanie trochę dłużej na środku planszy, co może pozwolić na szybsze 
przemieszczanie się pionków w przyszłości. Minusem tej strategii,
jest fakt, że pionki pozostające na planszy mogą pozwolić także przeciwnikowi
na szybsze przemieszczanie się. Kod źródłowy tej heurestyki jest 
bardzo podobny do poprzedniej, z tą rożnicą, że do ewaluacji każdego pionka 
odejmowana jest także odległość od środka planszy.
\begin{lstlisting}
float goalDistance = pawnHeuristic.evaluatePawnScore(
    row, col, goalPosition.first, goalPosition.second);
float centerDistance = pawnHeuristic.evaluatePawnScore(row, col, 8, 8);

evaluation -= squareDistancePenalty(goalDistance);
evaluation -= centerDistance;
\end{lstlisting}

\end{itemize}















\end{document}